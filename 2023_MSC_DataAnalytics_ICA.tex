% Options for packages loaded elsewhere
\PassOptionsToPackage{unicode}{hyperref}
\PassOptionsToPackage{hyphens}{url}
\PassOptionsToPackage{dvipsnames,svgnames,x11names}{xcolor}
%
\documentclass[
  letterpaper,
  DIV=11,
  numbers=noendperiod]{scrartcl}

\usepackage{amsmath,amssymb}
\usepackage{iftex}
\ifPDFTeX
  \usepackage[T1]{fontenc}
  \usepackage[utf8]{inputenc}
  \usepackage{textcomp} % provide euro and other symbols
\else % if luatex or xetex
  \usepackage{unicode-math}
  \defaultfontfeatures{Scale=MatchLowercase}
  \defaultfontfeatures[\rmfamily]{Ligatures=TeX,Scale=1}
\fi
\usepackage{lmodern}
\ifPDFTeX\else  
    % xetex/luatex font selection
\fi
% Use upquote if available, for straight quotes in verbatim environments
\IfFileExists{upquote.sty}{\usepackage{upquote}}{}
\IfFileExists{microtype.sty}{% use microtype if available
  \usepackage[]{microtype}
  \UseMicrotypeSet[protrusion]{basicmath} % disable protrusion for tt fonts
}{}
\makeatletter
\@ifundefined{KOMAClassName}{% if non-KOMA class
  \IfFileExists{parskip.sty}{%
    \usepackage{parskip}
  }{% else
    \setlength{\parindent}{0pt}
    \setlength{\parskip}{6pt plus 2pt minus 1pt}}
}{% if KOMA class
  \KOMAoptions{parskip=half}}
\makeatother
\usepackage{xcolor}
\setlength{\emergencystretch}{3em} % prevent overfull lines
\setcounter{secnumdepth}{-\maxdimen} % remove section numbering
% Make \paragraph and \subparagraph free-standing
\ifx\paragraph\undefined\else
  \let\oldparagraph\paragraph
  \renewcommand{\paragraph}[1]{\oldparagraph{#1}\mbox{}}
\fi
\ifx\subparagraph\undefined\else
  \let\oldsubparagraph\subparagraph
  \renewcommand{\subparagraph}[1]{\oldsubparagraph{#1}\mbox{}}
\fi


\providecommand{\tightlist}{%
  \setlength{\itemsep}{0pt}\setlength{\parskip}{0pt}}\usepackage{longtable,booktabs,array}
\usepackage{calc} % for calculating minipage widths
% Correct order of tables after \paragraph or \subparagraph
\usepackage{etoolbox}
\makeatletter
\patchcmd\longtable{\par}{\if@noskipsec\mbox{}\fi\par}{}{}
\makeatother
% Allow footnotes in longtable head/foot
\IfFileExists{footnotehyper.sty}{\usepackage{footnotehyper}}{\usepackage{footnote}}
\makesavenoteenv{longtable}
\usepackage{graphicx}
\makeatletter
\def\maxwidth{\ifdim\Gin@nat@width>\linewidth\linewidth\else\Gin@nat@width\fi}
\def\maxheight{\ifdim\Gin@nat@height>\textheight\textheight\else\Gin@nat@height\fi}
\makeatother
% Scale images if necessary, so that they will not overflow the page
% margins by default, and it is still possible to overwrite the defaults
% using explicit options in \includegraphics[width, height, ...]{}
\setkeys{Gin}{width=\maxwidth,height=\maxheight,keepaspectratio}
% Set default figure placement to htbp
\makeatletter
\def\fps@figure{htbp}
\makeatother

\KOMAoption{captions}{tableheading}
\makeatletter
\makeatother
\makeatletter
\makeatother
\makeatletter
\@ifpackageloaded{caption}{}{\usepackage{caption}}
\AtBeginDocument{%
\ifdefined\contentsname
  \renewcommand*\contentsname{Table of contents}
\else
  \newcommand\contentsname{Table of contents}
\fi
\ifdefined\listfigurename
  \renewcommand*\listfigurename{List of Figures}
\else
  \newcommand\listfigurename{List of Figures}
\fi
\ifdefined\listtablename
  \renewcommand*\listtablename{List of Tables}
\else
  \newcommand\listtablename{List of Tables}
\fi
\ifdefined\figurename
  \renewcommand*\figurename{Figure}
\else
  \newcommand\figurename{Figure}
\fi
\ifdefined\tablename
  \renewcommand*\tablename{Table}
\else
  \newcommand\tablename{Table}
\fi
}
\@ifpackageloaded{float}{}{\usepackage{float}}
\floatstyle{ruled}
\@ifundefined{c@chapter}{\newfloat{codelisting}{h}{lop}}{\newfloat{codelisting}{h}{lop}[chapter]}
\floatname{codelisting}{Listing}
\newcommand*\listoflistings{\listof{codelisting}{List of Listings}}
\makeatother
\makeatletter
\@ifpackageloaded{caption}{}{\usepackage{caption}}
\@ifpackageloaded{subcaption}{}{\usepackage{subcaption}}
\makeatother
\makeatletter
\@ifpackageloaded{tcolorbox}{}{\usepackage[skins,breakable]{tcolorbox}}
\makeatother
\makeatletter
\@ifundefined{shadecolor}{\definecolor{shadecolor}{rgb}{.97, .97, .97}}
\makeatother
\makeatletter
\makeatother
\makeatletter
\makeatother
\ifLuaTeX
  \usepackage{selnolig}  % disable illegal ligatures
\fi
\IfFileExists{bookmark.sty}{\usepackage{bookmark}}{\usepackage{hyperref}}
\IfFileExists{xurl.sty}{\usepackage{xurl}}{} % add URL line breaks if available
\urlstyle{same} % disable monospaced font for URLs
\hypersetup{
  pdftitle={Data Analytics for Immersive Environments - Individual Project},
  pdfauthor={Niall McGuinness},
  colorlinks=true,
  linkcolor={blue},
  filecolor={Maroon},
  citecolor={Blue},
  urlcolor={Blue},
  pdfcreator={LaTeX via pandoc}}

\title{Data Analytics for Immersive Environments - Individual Project}
\usepackage{etoolbox}
\makeatletter
\providecommand{\subtitle}[1]{% add subtitle to \maketitle
  \apptocmd{\@title}{\par {\large #1 \par}}{}{}
}
\makeatother
\subtitle{Interactive RDBMS data with Quarto \& Shiny}
\author{Niall McGuinness}
\date{2023-12-16}

\begin{document}
\maketitle
\ifdefined\Shaded\renewenvironment{Shaded}{\begin{tcolorbox}[boxrule=0pt, breakable, borderline west={3pt}{0pt}{shadecolor}, frame hidden, sharp corners, interior hidden, enhanced]}{\end{tcolorbox}}\fi

\renewcommand*\contentsname{Table of contents}
{
\hypersetup{linkcolor=}
\setcounter{tocdepth}{3}
\tableofcontents
}
\hypertarget{learning-outcomes}{%
\subsection{Learning Outcomes}\label{learning-outcomes}}

To practice the following:

\begin{itemize}
\tightlist
\item
  Writing and executing complex SQL queries to retrieve and manipulate
  data.
\item
  Understanding and applying SQL concepts in a database context.
\item
  Analysing and interpreting data from a SQL database effectively.
\item
  Developing and interpreting linear regression models in the context of
  game development data.
\item
  Understanding the relationship between different variables and how
  they impact each other.
\item
  Evaluating the performance of a linear regression model and
  understanding its limitations and applicability.
\item
  Utilising Quarto to create interactive Shiny web applications.
\item
  Designing user-friendly interfaces for data interaction and
  visualisation.
\item
  Implementing data visualisation techniques to effectively communicate
  data insights.
\item
  Applying data analytics techniques to real-world scenarios in game
  development.
\item
  Analysing and interpreting data specific to game development projects
  or products.
\item
  Demonstrating the ability to solve problems by applying analytical
  skills in a structured manner.
\end{itemize}

\hypertarget{functional-requirements}{%
\subsection{Functional Requirements}\label{functional-requirements}}

\hypertarget{part-a---sql-20-marks}{%
\subsubsection{Part A - SQL (20 Marks)}\label{part-a---sql-20-marks}}

You will be provided with an SQLite database file (via email before
\textbf{5PM 18th December 2023}) containing information about various
game development projects and their associated assets.

This database includes tables for projects, assets, developers,
timelines, and customers. The structure of each table is shown.

\hypertarget{projects}{%
\paragraph{\texorpdfstring{\textbf{Projects}}{Projects}}\label{projects}}

Stores information about each game development project. It includes
details like the project's unique identifier, name, start and end dates,
budget, and the status of the project. It also links to the customer
commissioning the project.

\begin{longtable}[]{@{}
  >{\raggedright\arraybackslash}p{(\columnwidth - 4\tabcolsep) * \real{0.1757}}
  >{\raggedright\arraybackslash}p{(\columnwidth - 4\tabcolsep) * \real{0.1486}}
  >{\raggedright\arraybackslash}p{(\columnwidth - 4\tabcolsep) * \real{0.6757}}@{}}
\toprule\noalign{}
\begin{minipage}[b]{\linewidth}\raggedright
Column Name
\end{minipage} & \begin{minipage}[b]{\linewidth}\raggedright
Data Type
\end{minipage} & \begin{minipage}[b]{\linewidth}\raggedright
Description
\end{minipage} \\
\midrule\noalign{}
\endhead
\bottomrule\noalign{}
\endlastfoot
ProjectID & Integer & Unique identifier for each project (Primary
Key) \\
ProjectName & String & Name of the game development project \\
StartDate & Date & Start date of the project \\
EndDate & Date & End date of the project \\
Budget & Decimal & Total budget allocated for the project \\
Status & String & Current status of the project (e.g., In Progress,
Completed, Cancelled) \\
CustomerID & Integer & Identifier linking to the Customers table
(Foreign Key) \\
\end{longtable}

\hypertarget{assets}{%
\paragraph{\texorpdfstring{\textbf{Assets}}{Assets}}\label{assets}}

Stores information about the various assets used in game development
projects. Each asset has a unique identifier, and the table includes
information about the project it is associated with, the name and type
of the asset, and its creation date.

\begin{longtable}[]{@{}
  >{\raggedright\arraybackslash}p{(\columnwidth - 4\tabcolsep) * \real{0.1842}}
  >{\raggedright\arraybackslash}p{(\columnwidth - 4\tabcolsep) * \real{0.1447}}
  >{\raggedright\arraybackslash}p{(\columnwidth - 4\tabcolsep) * \real{0.6711}}@{}}
\toprule\noalign{}
\begin{minipage}[b]{\linewidth}\raggedright
Column Name
\end{minipage} & \begin{minipage}[b]{\linewidth}\raggedright
Data Type
\end{minipage} & \begin{minipage}[b]{\linewidth}\raggedright
Description
\end{minipage} \\
\midrule\noalign{}
\endhead
\bottomrule\noalign{}
\endlastfoot
AssetID & Integer & Unique identifier for each asset (Primary Key) \\
ProjectID & Integer & Identifier linking to the Projects table (Foreign
Key) \\
AssetName & String & Name of the asset \\
Type & String & Type of the asset (e.g., 3D Model, Animation,
Texture) \\
CreationDate & Date & Date when the asset was created \\
\end{longtable}

\hypertarget{developers}{%
\paragraph{\texorpdfstring{\textbf{Developers}}{Developers}}\label{developers}}

Stores information about the developers working on various projects. It
includes each developer's unique identifier, name, area of
specialization, and the number of years of experience they have in the
field.

\begin{longtable}[]{@{}
  >{\raggedright\arraybackslash}p{(\columnwidth - 4\tabcolsep) * \real{0.2133}}
  >{\raggedright\arraybackslash}p{(\columnwidth - 4\tabcolsep) * \real{0.1467}}
  >{\raggedright\arraybackslash}p{(\columnwidth - 4\tabcolsep) * \real{0.6400}}@{}}
\toprule\noalign{}
\begin{minipage}[b]{\linewidth}\raggedright
Column Name
\end{minipage} & \begin{minipage}[b]{\linewidth}\raggedright
Data Type
\end{minipage} & \begin{minipage}[b]{\linewidth}\raggedright
Description
\end{minipage} \\
\midrule\noalign{}
\endhead
\bottomrule\noalign{}
\endlastfoot
DeveloperID & Integer & Unique identifier for each developer (Primary
Key) \\
Name & String & Name of the developer \\
Specialisation & String & Area of specialisation (e.g., Programmer,
Animator, Artist) \\
ExperienceYears & Integer & Number of years of experience in the
field \\
\end{longtable}

\hypertarget{timelines}{%
\paragraph{\texorpdfstring{\textbf{Timelines}}{Timelines}}\label{timelines}}

Tracks significant milestones for each project. It includes a unique
identifier for each timeline entry, links to the respective project,
descriptions of milestones, and both expected and actual completion
dates.

\begin{longtable}[]{@{}
  >{\raggedright\arraybackslash}p{(\columnwidth - 4\tabcolsep) * \real{0.2593}}
  >{\raggedright\arraybackslash}p{(\columnwidth - 4\tabcolsep) * \real{0.1358}}
  >{\raggedright\arraybackslash}p{(\columnwidth - 4\tabcolsep) * \real{0.6049}}@{}}
\toprule\noalign{}
\begin{minipage}[b]{\linewidth}\raggedright
Column Name
\end{minipage} & \begin{minipage}[b]{\linewidth}\raggedright
Data Type
\end{minipage} & \begin{minipage}[b]{\linewidth}\raggedright
Description
\end{minipage} \\
\midrule\noalign{}
\endhead
\bottomrule\noalign{}
\endlastfoot
TimelineID & Integer & Unique identifier for each timeline entry
(Primary Key) \\
ProjectID & Integer & Identifier linking to the Projects table (Foreign
Key) \\
Milestone & String & Description of the milestone \\
ExpectedCompletionDate & Date & Expected date of completion for the
milestone \\
ActualCompletionDate & Date & Actual date of completion for the
milestone \\
\end{longtable}

\hypertarget{customers}{%
\paragraph{\texorpdfstring{\textbf{Customers}}{Customers}}\label{customers}}

Stores information about the clients commissioning game development
projects. It includes a unique identifier for each customer, along with
their name, city, and country.

\begin{longtable}[]{@{}
  >{\raggedright\arraybackslash}p{(\columnwidth - 4\tabcolsep) * \real{0.2237}}
  >{\raggedright\arraybackslash}p{(\columnwidth - 4\tabcolsep) * \real{0.1447}}
  >{\raggedright\arraybackslash}p{(\columnwidth - 4\tabcolsep) * \real{0.6316}}@{}}
\toprule\noalign{}
\begin{minipage}[b]{\linewidth}\raggedright
Column Name
\end{minipage} & \begin{minipage}[b]{\linewidth}\raggedright
Data Type
\end{minipage} & \begin{minipage}[b]{\linewidth}\raggedright
Description
\end{minipage} \\
\midrule\noalign{}
\endhead
\bottomrule\noalign{}
\endlastfoot
CustomerID & Integer & Unique identifier for each customer (Primary
Key) \\
CustomerName & String & Name of the customer \\
CustomerCity & String & City of the customer \\
CustomerCountry & String & Country of the customer \\
\end{longtable}

\hypertarget{projectdevelopers}{%
\paragraph{\texorpdfstring{\textbf{ProjectDevelopers}}{ProjectDevelopers}}\label{projectdevelopers}}

Links developers to the projects they are working on and shows which
developer worked on which project and their specific role in that
project.

\begin{longtable}[]{@{}
  >{\raggedright\arraybackslash}p{(\columnwidth - 4\tabcolsep) * \real{0.1807}}
  >{\raggedright\arraybackslash}p{(\columnwidth - 4\tabcolsep) * \real{0.1325}}
  >{\raggedright\arraybackslash}p{(\columnwidth - 4\tabcolsep) * \real{0.6867}}@{}}
\toprule\noalign{}
\begin{minipage}[b]{\linewidth}\raggedright
Column Name
\end{minipage} & \begin{minipage}[b]{\linewidth}\raggedright
Data Type
\end{minipage} & \begin{minipage}[b]{\linewidth}\raggedright
Description
\end{minipage} \\
\midrule\noalign{}
\endhead
\bottomrule\noalign{}
\endlastfoot
ProjectID & Integer & Identifier linking to the Projects table (Foreign
Key) \\
DeveloperID & Integer & Identifier linking to the Developers table
(Foreign Key) \\
Role & String & Role of the developer in the project (e.g., Lead,
Contributor) \\
\end{longtable}

\hypertarget{assetsdevelopers}{%
\paragraph{\texorpdfstring{\textbf{AssetsDevelopers}}{AssetsDevelopers}}\label{assetsdevelopers}}

Links which developer was the principal developer on each asset.

\begin{longtable}[]{@{}
  >{\raggedright\arraybackslash}p{(\columnwidth - 4\tabcolsep) * \real{0.1899}}
  >{\raggedright\arraybackslash}p{(\columnwidth - 4\tabcolsep) * \real{0.1392}}
  >{\raggedright\arraybackslash}p{(\columnwidth - 4\tabcolsep) * \real{0.6709}}@{}}
\toprule\noalign{}
\begin{minipage}[b]{\linewidth}\raggedright
Column Name
\end{minipage} & \begin{minipage}[b]{\linewidth}\raggedright
Data Type
\end{minipage} & \begin{minipage}[b]{\linewidth}\raggedright
Description
\end{minipage} \\
\midrule\noalign{}
\endhead
\bottomrule\noalign{}
\endlastfoot
AssetID & Integer & Identifier linking to the Assets table (Foreign
Key) \\
DeveloperID & Integer & Identifier linking to the Developers table
(Foreign Key) \\
\end{longtable}

First, you are required to write \textbf{three} core SQL queries to
perform the following tasks:

\begin{enumerate}
\def\labelenumi{\arabic{enumi}.}
\tightlist
\item
  List the total budget allocated for projects in each country, along
  with the count of projects per country. Display sorted by the total
  budget in descending order.
\item
  List the average development time for projects, categorized by the
  number of assets used.
\item
  List the top three developers based on the number of successful
  projects they've been involved in. Display the results.
\end{enumerate}

Next, you are also required to demonstrate the following \textbf{three}
general SQL concepts using no fewer than \textbf{three} distinct SQL
statements:

\begin{enumerate}
\def\labelenumi{\arabic{enumi}.}
\tightlist
\item
  SELECT with LIKE and OR
\item
  SELECT with DISTINCT and ORDER BY
\item
  Subquery with SELECT
\end{enumerate}

Display the results in formatted table(s) in a Quarto Notebook file
inside a \textbf{Quarto Project}. You should add pagination to the
tables if the number of rows is larger than 8-10 (see Recommended
Reading on creating tables)

\hypertarget{part-b---linear-regression-20-marks}{%
\subsubsection{Part B - Linear Regression (20
Marks)}\label{part-b---linear-regression-20-marks}}

Utilise the provided dataset, which contains details about project
budgets, timelines, team sizes, and success rates and apply linear
regression analysis to understand trends or patterns in the game
development lifecycle.

You are required to perform the following tasks:

\begin{enumerate}
\def\labelenumi{\arabic{enumi}.}
\tightlist
\item
  \textbf{Model}: Perform linear regression to predict the success rate
  of a project based on its budget \textbf{and} team size and present
  the data in an appropriate plot.
\item
  \textbf{Interpret}: Interpret the model coefficients and discuss what
  insights they provide about game development.
\item
  \textbf{Discuss}: Comment on the reliability of the the linear
  regression model and any outliers present in the data.
\end{enumerate}

Output the results in a Quarto Notebook file inside a \textbf{Quarto
Project}.

\hypertarget{part-c---interactive-quarto-dashboard-shiny-web-application-30-marks}{%
\subsubsection{Part C - Interactive Quarto Dashboard \& Shiny Web
Application (30
Marks)}\label{part-c---interactive-quarto-dashboard-shiny-web-application-30-marks}}

Create an interactive web application using Quarto Dashboards and Shiny
to display and interact with data from the database. You are required
develop an interactive web web application containing \textbf{two
pages/tabs} which allow the following:

\begin{enumerate}
\def\labelenumi{\arabic{enumi}.}
\tightlist
\item
  \textbf{Data View}: The first page/tab should allow the user to view
  the contents of \textbf{any four tables} in the database with no
  filtering or queries executed.You may need to consider using a more
  advanced table library (e.g.~kable + kableExtra, reactable, DT) to add
  pagination to larger tables.
\item
  \textbf{Plot View}: The second page/tab should allow users to
  visualize data through 2-3 appropriate plots and \textbf{interact}
  with \textbf{each} plot. Typically, plots could include data related
  to project timelines, budget, asset types, etc.
\end{enumerate}

The second page/tab must support user interactions via input fields
(e.g., textfield, dropdown, slider, checkbox) to allow interactions such
as: * Selecting developers to view more detailed data on the projects
they participated in. * Specifying a completion date range for projects.

You must include a minimum of \textbf{two fields} for interaction for
\textbf{each} plot.

Output the results in a properly a \textbf{Quarto Project}. The web
application does \textbf{not} need to be published to a remote server
and may be run from your machine.

\hypertarget{part-d---quality-conclusions-20-marks}{%
\subsubsection{Part D - Quality \& Conclusions (20
Marks)}\label{part-d---quality-conclusions-20-marks}}

You are required to present Assess the student's ability to present a
Quarto project containing three well-formatted documents. You will be
assessed in terms of adherence to best practices in coding and data
visualization and the quality of your conclusion. The assessment
criteria for this component are listed below:

\hypertarget{document-formatting-4-marks}{%
\paragraph{Document Formatting (4
Marks)}\label{document-formatting-4-marks}}

\begin{itemize}
\tightlist
\item
  Overall structure and organization of the report.
\item
  Effective use of headings, subheadings, and bullet points.
\item
  Consistency in font, spacing, and margins.
\item
  Proper pagination and table of contents (if applicable).
\end{itemize}

\hypertarget{suitability-of-tables-in-part-a-4-marks}{%
\paragraph{Suitability of Tables in Part A (4
Marks)}\label{suitability-of-tables-in-part-a-4-marks}}

\begin{itemize}
\tightlist
\item
  Clarity and readability of tables.
\item
  Appropriate labeling for tables.
\item
  Consistent and legible formatting in tables and figures.
\end{itemize}

\hypertarget{suitability-of-plots-in-part-b-4-marks}{%
\paragraph{Suitability of Plots in Part B (4
Marks)}\label{suitability-of-plots-in-part-b-4-marks}}

\begin{itemize}
\tightlist
\item
  Relevance and effectiveness of chosen plots to represent the data.
\item
  Ability to convey the intended message or findings clearly.
\item
  Innovative and insightful use of data visualization techniques.
\end{itemize}

\hypertarget{adherence-to-best-practices-in-coding-and-data-visualization-5-marks}{%
\paragraph{Adherence to Best Practices in Coding and Data Visualization
(5
Marks)}\label{adherence-to-best-practices-in-coding-and-data-visualization-5-marks}}

\begin{itemize}
\tightlist
\item
  Code clarity, including commenting and organization.
\item
  Use of efficient and appropriate coding methods.
\item
  Adherence to data visualization best practices, including color
  choice, scale, and avoiding misleading representations.
\end{itemize}

\hypertarget{conclusion-and-reflection-4-marks}{%
\paragraph{Conclusion and Reflection (4
Marks)}\label{conclusion-and-reflection-4-marks}}

\begin{itemize}
\tightlist
\item
  Inclusion of a 1-2 paragraph Conclusion section.
\item
  Depth of reflection on what was learned from each part of the
  assignment.
\item
  Insightful discussion on how the skills and knowledge gained apply to
  game development and computer animation.
\end{itemize}

\hypertarget{part-e---technology-exploration-10-marks}{%
\subsubsection{Part E - Technology Exploration (10
Marks)}\label{part-e---technology-exploration-10-marks}}

Finally, you will be awarded a maximum of 10 marks for creative use of R
packages which have \textbf{not} been covered in class. For example,
this could be a package for rendering tables (i.e.~reactable), animating
plots (i.e.~gganimate), or rendering the geospatial data from the
database (i.e.~customer city and customer country) using additional R
packages (i.e.~ggplot2 and ggmap).

Note, in order to receive the marks for this component you should
include a short paragraph in a separate Quarto Markdown file, entitled
\emph{Technology Exploration.qmd}, which summarises what component(s)
you have added.

\hypertarget{version-control-requirements}{%
\subsection{Version Control
Requirements}\label{version-control-requirements}}

You must use a recognised online code repository (e.g., GitHub) and make
regular well-named commits to your private repository. A link to your
code repo must be included in a README as part of the final submission
and you must add your lecturer as a developer to the repository. The
repository must be named \emph{2023\_DAIE\_ICA\_StudentInitials}.

Your grade for this component will depend directly upon the
\textbf{regularity} of your commits. A development project of this size
should consist of a minimum of \textbf{5+ distinct commit messages}
spread over the lifetime of the development. Committing all your code in
one commit, before the deadline, will be interpreted negatively.

\textbf{Any submission made which does not include a repository link
will not be graded.}

\hypertarget{submission-requirements}{%
\subsection{Submission Requirements}\label{submission-requirements}}

\begin{enumerate}
\def\labelenumi{\arabic{enumi}.}
\tightlist
\item
  A link to your R source code repository. Ensure that \textbf{no}
  changes are made to the repository following the submission deadline.
  You should create a separate fork for this submission and leave it
  unchanged after the deadline. Ask your lecturer for details on fork
  creation.
\item
  The assignment must be entirely the work of each student. Students are
  \textbf{not} permitted to share any pseudocode or source code from
  their solution with any other group in the class.
\item
  Students may \textbf{not} distribute the source code of their solution
  to any other student, in any format (i.e., electronic, verbal, or
  hardcopy transmission).
\item
  Plagiarised assignments will receive a mark of \textbf{zero}. This
  also applies to the individual/group allowing their work to be
  plagiarised. Any plagiarism will be reported to the Head of Department
  and a report will be added to your permanent academic record.
\item
  Late assignments will \textbf{only} be accepted if accompanied by the
  appropriate medical note. This documentation must be received within
  10 working days of the project deadline. The Institute standard
  penalties for late submission will apply.
\item
  Each student \textbf{must} complete and \textbf{sign} a single
  assignment cover sheet. Please submit the signed cover sheet before 5
  pm on the Friday of the week of the deadline.
\end{enumerate}

\hypertarget{recommended-reading}{%
\subsection{Recommended Reading}\label{recommended-reading}}

\begin{itemize}
\tightlist
\item
  \href{https://www.openintro.org/book/os/}{Recommended Text - OpenIntro
  Statistics Videos \& Slides}
\item
  \href{https://www.w3schools.com/sql/}{SQL Tutorial}
\item
  \href{https://blog.rsquaredacademy.com/working-with-databases-using-r/}{A
  Comprehensive Introduction to Working with Databases using R}
\item
  \href{https://rfortherestofus.com/2019/11/how-to-make-beautiful-tables-in-r\#:~:text=kable\%20\%2B\%20kableExtra\&text=The\%20goal\%20of\%20kableExtra\%20is,similar\%20with\%20ggplot2\%20and\%20plotly\%20.}{How
  to Make Beautiful Tables in R}
\item
  \href{https://www.scribbr.com/statistics/simple-linear-regression/}{Simple
  Linear Regression \textbar{} An Easy Introduction \& Examples}
\item
  \href{https://youtu.be/-mGXnm0fHtI}{Linear regression using R
  programming}
\item
  \href{https://www.geeksforgeeks.org/add-regression-line-to-ggplot2-plot-in-r/}{Add
  Regression Line to ggplot2 Plot in R}
\item
  \href{https://youtu.be/1Vs-ckEI94M}{Fitting and visualizing linear
  regression models with the ggplot2 R package (CC237)}
\item
  \href{https://youtu.be/HPJn1CMvtmI\%5D}{ggplot for plots and graphs.
  An introduction to data visualization using R programming}
\item
  \href{https://www.youtube.com/watch?v=_VGJIPRGTy4}{Quarto Dashboards
  \textbar{} Charles Teague \textbar{} Posit}
\item
  \href{https://quarto.org/docs/dashboards/interactivity/shiny-r.html}{Dashboards
  with Shiny for R}
\end{itemize}

\hypertarget{plotting-resources}{%
\subsection{Plotting Resources}\label{plotting-resources}}

\begin{itemize}
\tightlist
\item
  \href{https://youtu.be/jX4_Dnzhl0M}{How to style your Quarto docs
  without knowing HTML \& CSS}
\item
  \href{https://youtu.be/FWLxE-ARuO8}{Drag-and-drop ggplot2 graphs with
  the Esquisse library}
\item
  \href{https://youtu.be/rfR9Nrpfnyg}{Visualize your data using ggplot.
  R programming is the best platform for creating plots and graphs.}
\item
  \href{https://www.youtube.com/watch?v=SnCi0s0e4Io}{Animate Graphs in
  R: Make Gorgeous Animated Plots with gganimate}
\end{itemize}

\hypertarget{additional-resources}{%
\subsection{Additional Resources}\label{additional-resources}}

The resources below are provided as a mixture of \textbf{optional
background reading}, technical reference, and useful tutorials.

\begin{itemize}
\tightlist
\item
  \href{https://youtu.be/Qrz2rUWM-uY}{YouTube - Hello, Quarto: A World
  of Possibilities (for Reproducible Publishing)}
\item
  \href{https://youtu.be/hbf7Ai3jnxY?t=1693}{YouTube - Beautiful reports
  and presentations with Quarto}
\item
  \href{https://quarto.org/docs/reference/formats/opml.html}{OPML
  Options}
\end{itemize}



\end{document}
